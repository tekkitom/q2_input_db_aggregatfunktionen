\documentclass[aspectratio=169]{beamer}
% \documentclass{beamer}
\setbeamertemplate{footline}{%
  \begin{beamercolorbox}[wd=\paperwidth,ht=2.25ex,dp=1ex]{date in
head/foot}%
    \hspace*{1ex} \insertframenumber{} / \inserttotalframenumber
  \end{beamercolorbox}}%
\setbeamertemplate{navigation symbols}{}

\mode<presentation>
{
   \useoutertheme{infolines}
%    \usetheme{Singapore}
  \usetheme{Boadilla}
%  \usetheme[secheader]{Boadilla}
    \useinnertheme[shadow]{rounded}
%   \subject{Informatik}
}

% \mode<handout>
% {
%   \usetheme{Montpellier}
%   \usecolortheme{dove}
% %   \usepackage{pgfpages}
%   \pgfpagesuselayout{4 on 1}[a4paper,landscape,border shrink=5mm]
% }


\usepackage[german]{babel}
\usepackage[utf8]{inputenc}
\usepackage{times}
\usepackage[T1]{fontenc}
\usepackage{eurosym}
\usepackage{graphicx}
\usepackage{ulem}
\usepackage{listings}
\usepackage{beamerfoils}
\lstset{numbers=left, numberstyle=\tiny, stepnumber=2, numbersep=5pt, language = html}
% \usepackage[markup=nocolor,deletedmarkup=xout]{changes}

\title{Aggregatfunktionen in SQL}

\author{Thomas Maul}
\institute[BWS Hofheim]{Brühlwiesenschule, Hofheim}

% \date[09.11.22] % (optional)
% {09. November 22}

 \MyLogo{\includegraphics[height=1cm]{bwslogo_3.png}}
% \MyLogo{\includegraphics[height=1cm]{../../../bilder/bwslogo_3.png}}


\begin{document}

\begin{frame}
  \titlepage
 % \hyperlink{Teil_2}{\beamerbutton{Go part 2}}
\end{frame}
\begin{frame}
  \frametitle{Statistik}
  \begin{itemize}
    \item Summe einer Spalte
    \item Mittelwert, kleinster, größter Wert
    \item Anzahl
  \end{itemize}
\end{frame}

\begin{frame}
  \frametitle{Aggregatfunktionen}
  \begin{description}
    \item[sum] Summe aller Werte: select sum(Spalte) \dots
    \item[Kleinster] select min(Spalte) \dots;
    \item[Größter] select max(Spalte) \dots;
    \item[Anzahl] Anzahl der Zeilen: select count(Spalte) \dots
  \end{description}
  Ergebnis ist immer eine Zeile
\end{frame}

\begin{frame}
  \frametitle{und was soll \dots\  bedeuten?}
  \begin{itemize}
    \item Eingrenzen mit where
    \item Group by
  \end{itemize}
\end{frame}

  
\end{document}

